\section{Fluorescence Life-Time Measurement (FLIM)}

\section{Lebenszeit von CFP und YFP}

In diesem Versuchsteil geht es um die Bestimmung der Lebensdauer des Donors und des Akzeptors. Dies tut man am 
besten indem man eine einfache Exponentialfunktion an die Daten anpasst. Dabei passt man zwei Parameter an, die Amplitude $A_0$ und 
Lebensdauer $\tau$.

\begin{equation}
    A(t) = A_0 \cdot \exp(-\frac{t}{\tau})
\end{equation}

Aus den einzelnen Parametern der angepassten Exponentialfunktionen bildet man einen Mittelwert. Das es jeweils für CFP und YFP 9 Werte gibt, 
verkleinert sich der Fehler um einen Faktor drei.\\
Aus den Anpassungen erhält man folgende Werte:\\

\begin{center}
    \centering
    \begin{tabular}{l|cr}
        \toprule
        Messung & Lebensdauer CFP (ns)& Lebensdauer YFP (ns)\\
        \midrule
        Kanal 1 & 2,4985& 3,0513\\
        Kanal 1 & 2,5074& 3,0670\\
        Kanal 1 & 2,5623& 3,0482\\
        \midrule
        Kanal 2 & 2,5483& 3,0482\\
        Kanal 2 & 2.4853& 3,1066\\
        Kanal 2 & 2,5123& 3,0923\\
        \midrule
        Kanal 1\&2 & 2,5458& 3,0923\\
        Kanal 1\&2 & 2,5063& 3,0682\\
        Kanal 1\&2 & 2,5595& 3,0500\\
        \midrule
        Mittelwert & 2,5251& 3,0693\\
        Fehler& 0,0036& 0,0043\\
        Ergebnis:& \textcolor{red}{2,5251 $\pm$ 0,0012}& \textcolor{red}{3,0693 $\pm$ 0,0043}\\
        \bottomrule
    \end{tabular}
\end{center}

Man sieht also einen Unterschied in der Lebenszeit der beiden Proteine. Der eigentlich spannende Teil 
ist jedoch die Lebensdauer des Donators mit und ohne Fret zu vergleichen.

\section{Lebensdauer mit FRET}

Man erwartet eine Verkürzung der Lebensdauer $\tau_D$ des Donator bei hinzukommen von FRET. Dies geschieht 
- wie in den Grundlagen erklärt - aufgrund der Kopplung an den Akzeptor. Um es vorweg zu nehmen. Die Methode funktioniert hier sehr schlecht. 
Man versucht das zu bewerkstelligen indem man eine Anpassung zweier Exponentialfunktionen macht. Diese beinhalten dann insgesamt 4 Parameter; 
ein Fit mit 4 Parametern ist jedoch sehr Störungsanfällig.