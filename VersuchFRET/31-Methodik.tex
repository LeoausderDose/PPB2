% Leonhard Schatt

\chapter{Methodik}

\section{Aufbau}

\subsection{Versuchsanordnung}

Verwendet wird in konfokales Laser-Scanning-Mikroskop (Leica SP5 I I ) mit einer Pico Harp
300-Einheit zur Durchführung von zeitkorrelierten Einzelphotonenmessungen. Bei Anregen verwendet man einen Argon-Ionen-Laser 
und einen gepulsten 470nm-Laser. Auf dem Mikroskop befindet sich ein Öl-Immersionsobjektiv (63x Vergröÿerung, NA=1,40). Detektiert wird 
mit einem Photomultiplier (PMT) mit variabler spektraler Selektion (\cite{FRETSkript}). 
Genaueres kann man im angehängten Protokoll nachschlagen.

\subsection{Aufnahmen der Sensitized Emission}

Um FRET nachzuweisen muss man bei diesem Versuchsteil mehrere Kanäle betrachten. Dazu nimmt man Bilder in Kanal 1/2 
bei 470-500\,nm/520-550\,nm und dies von einer YFP, einer CFP und einer YFP/CFP Probe. Dabei wird immer drei Bilder 
genommen: Anregung und Detektion des Donor, Anregung und Detektion des Akzeptors und Anregung des Donors und Detektion des Akzeptors.


\subsection{Donoremission nach Akzeptorbleichen}

In diesem Versuchsteil werden die Proben jeweils 12 Frames gebleicht danach wir mit Kanal 1 und 2 eine Bilderreihe 
aufgenommen und vor und nach der Bleichung. Danach wird ein Areal ausgewählt (siehe Abb. \ref{bild:bleachROI}) und aus 
diesem werden wir dann die Emissionsraten auftragen und exportieren. 

\subsection{Lebensdauermessungen}

Zuerst misst man an CFP und YFP die Lebenszeit. Dies tut man indem man mit dem gepulsten 470nm Laser bei 40 MHz anregt und dann die Zeit misst, bis ein Photon zurück kommt. 
Wenn man die Lebensdauer dann in ein Histogramm aufträgt, kann man dies mit dem Programm fitten.\\
Danach tut man das selbe bei der CFP/YFP Probe. Bei nimmt man sowohl auf Kanal 1 und 2 auf.