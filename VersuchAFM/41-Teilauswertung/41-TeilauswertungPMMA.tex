\clearpage
\section{PS/PMMA}

In diesem Teil der Auswertung wird demonstriert, warum die Information über die Phase beim AFM sehr aufschlussreich sein kann. Hier haben wir 
als Probe ein gemisch aus Polystyrol und Polymethylmethacrylat. Diese wurden vermischt und dann auf einem Silizium-Waver aufgebracht. 
Jetzt fragt man sich im nachhinein, welches Material welches ist. Aus den topographischen Daten ist das leider nicht mehr ersichtlich. Abgesehen davon kann man auch 
nicht so gut unterscheiden, welcher Teil des Gemisches wo endet (siehe Abbildung \ref{PMMA1}.)\\
\begin{figure}[h]
    \centering
    \includegraphics[width = 9cm]{Bilder/PMMA/PSPMMA.jpg}
    \caption{PS/PMMA in der Topographieansicht}
    \label{PMMA1}
\end{figure}

Betrachtet man nun aber das Phasenbild fällt auf, dass die Unterschied eindeutig zu sehen sind. Dabei gibt es sehr dunkle und helle Punkte 
in der Abbildung \ref{PMMA2}.\\

\begin{figure}
    \centering
    \includegraphics[width = 9cm]{Bilder/PMMA/PSPMMAPhase.jpg}
    \caption{Phaseninformation zur selben Aufnahmen wie in Abbildung \ref{PMMA1}}
    \label{PMMA2}
\end{figure}

Das liegt daran, dass beim Phasenkontrastmodus Informationen über die lokale Härte der Probe gesammelt werden. Man kann also 
Rückschlüsse auf die chemische Zusammensetzung beispielsweise machen, welche rein aus dem topographischen Daten nicht ersichtlich waren.\\
Dabei wird der Phasenkontrastmodus vor allem verwendet um einen guten Kontrast im Bild zu erzeugen und qualitative Aussagen zu machen. 
Für quantitative Aussagen ist er oft zu ungenau, da er von zu vielen oft nur ungefähr bekannten Größen wie Federkonstante, unterschiedlichen Kräften und Spitzengeometrie 
abhängt (\cite[S.68]{SampleKit2007}).
Hier kann man beispielweise sagen, dass die hellen Flecken Polystyrol sind, das dieses härter ist als Polymethylmethacrylat\footnotemark.
\footnotetext{\url{https://de.wikipedia.org/wiki/Polymethylmethacrylat}, Eingesehen am 22.09.2021}