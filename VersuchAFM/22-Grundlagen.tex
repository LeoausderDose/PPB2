% Matteo Kumar
% PPB2 Solarzelle
%Grundlagen

\chapter{Grundlagen}
\section{Aufbau und Funktionsweise}
Im Gegensatz zu optischen Mikroskopen basiert die Rasterkraftmikroskopie (atomic force mikroscopy, kurz AFM) auf der Detektion von Kräften. 
Hauptbestandteil sind eine Feder (Cantilever) und eine sehr feine, im Idealfall einatomige, Spitze. Mit dieser wird die Probe abgerastert 
(wie genau, siehe \ref{sec:Modi}). Dabei existieren eine Fülle an anziehenden und abstoßenden Kräften zwischen Probe und Spitze, beispielsweise 
Londonsche Dispersionskräfte oder Coulomb-Wechselwirkungen. Die Gesamtheit dieser Wechselwirkungen kann mit dem Lennard-Jones-Potential 
angenähert werden. Dieses ist, in Abhängigkeit vom Abstand $r$ zwischen den Molekülen, mit molekülabhängigen Parametern $a$, $b$: 
\begin{equation*}
    \Phi_{LJ}(r) = \frac{a}{r^{12}} - \frac{b}{r^6}
\end{equation*}
(\cite{Demtroeder2013}, S.126f) \\
Eine Skizze des Potentials ist in Abb. \ref{bild:LJP} zu sehen.

\begin{figure}[h]
    \centering
    \includegraphics[scale = 0.135]{Bilder/LennardJones.jpg}
    \caption{Skizze des Lennard-Jones-Potentials \protect \footnotemark}
    \label{bild:LJP}
\end{figure}
\footnotetext{\cite{Haefner2019}, S.94}

Diese Wechselwirkungen üben eine Kraft auf den Cantilever aus, der sich infolgedessen verbiegt. Diese Verbiegung wird über einen 
Laserstrahl detektiert, der auf den Cantilever gerichtet ist. Verbiegt sich dieser, so wird auch der Strahl des Lasers unter einem anderen 
Winkel abgelenkt, was ausgewertet werden kann (\cite{Haugstad2012}, S.5). Ein schematischer Aufbau ist in Abb. \ref{bild:Aufbau} zu sehen.

\begin{figure}[h]
    \centering
    \includegraphics[scale = 0.45]{Bilder/AufbauAFM.png}
    \caption{Schematischer Aufbau eines Rasterkraftmikroskops \protect \footnotemark}
    \label{bild:Aufbau}
\end{figure}
\footnotetext{\cite{easyScan2003}, S.6}

Duch das Funktionsprinzip über die Detektion von wechselwirkungen sind bei der AFM auch keine besonderen Anforderungen an die Probe 
gestellt. Sie müssen insbesondere nicht leitfähig oder im Vakuum sein. \footnotemark \, Dennoch sollten sie frei vor Verunreinigungen sein 
und vor Feuchte geschützt werden; zudem sollte die Probe zu der verwendeten Spitze passen. \\
\footnotetext{Eine AFM im Vakuum liefert dennoch bessere Ergebnisse; für die meisten Zwecke ist eine Untersuchung an der Luft aber 
vollkommen ausreichend}


\newpage


\section{Verwendete Modi}
\label{sec:Modi}
Bei der AFM gibt es einige Betriebsmodi; die im Versuch verwendeten sollen hier kurz erläutert werden.

\subsection{Contact Mode}
Im Contact Mode hat die Spitze des Cantilevers dauerhaft Kontakt mit der zu untersuchenden Probe. Man unterscheidet dabei wiederum zwei Modi: \\
Der Constant Hight Mode ist der einfachster aller Modi. In diesem wird der Cantilever auf einer konstanten Höhe über die Probe gefahren. 
Eine Regelung ist nicht nötig. (\url{https://www.nanosurf.com/en/support/afm-modes-overview/contact-modes}, Stand: 23.09.21) \\
Im Constant Force Mode wird nicht die Höhe über der Probe, sondern die Kraft, die auf die Spitze wirkt, konstant gehalten. Diese wird 
über den sog. Setpoint eingestellt. Zur Beibehaltung der Kraft ist ein PID-Regler notwendig. (\cite{Rieger2013}, S.8)\\
Obwohl der Contact Mode leicht zu realisieren ist, hat er einige Nachteile. So ist durch den dauerhaften Kontakt die Abnutzung der Spitze 
vergleichsweise hoch; zudem kann diese bei plötzlichen Erhebungen oder schlecht gewählten Einstellungen leicht abbrechen. (\cite{Vesely2017}, S.26)

\subsection{Non-Contact Mode}
Im Non-Contact- oder Tapping-Mode (je nachdem ob die Spitze die Probe nie berührt oder kurz antippt) wird der Cantilever mit einer 
konstanten Frequenz nahe seiner Resonanzfrequenz zum Schwingen angeregt. Diese wird wieder über einen PID-Regler gehalten. 
Die Wechselwirkungen zwischen Probe und Spitze bewirken zum einen eine Änderung der Schwingungsamplitude, über die topographische 
Erkenntnisse gewonnen werden können. (\cite{Vesely2017}, S.26). Zum anderen ergibt sich auch eine Phasenverschiebung, die dazu genutzt 
werden kann, Eigenschaften der Probe zu bestimmen (Dichte etc.). (\url{https://www.nanosurf.com/en/support/afm-modes-overview/dynamic-modes}, Stand: 23.09.21)\\
