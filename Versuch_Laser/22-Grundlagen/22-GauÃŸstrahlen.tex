\section{Gaußstrahlen}
\label{section:Gaussian}

Das optische Konzept des Gaußstrahls zeichnet sich dadurch aus, dass der entsprechende Lichtstrahl, 
wenn man ihn senkrecht zur Ausbreitungsrichtung schneidet, ein Gaußprofil aufweist.
Dieses ist jedoch in der Weite des Gaußstrahles nicht konstant. Der Strahl hat eine 
Strahltaille wie in Abbildung \ref{bild:taille} zu sehen.

\begin{figure}[ht]
    \centering
    \includegraphics[width = 10cm]{Bilder/Auswertung/GAußtaille.png}
    \caption{Strahltaille eines Gaußstrahls. Bild ist ein Querschnitt durch den Gaußstrahl entlang der Ausbreitungsrichtung, 
    Bild von Aleph, \url{http://commons.wikimedia.org}}
    \label{bild:taille}
\end{figure}

\subsection{Strahlausbreitungsfaktor}

Der Strahlausbreitungsfaktor kommt aus dem Strahlparameterprodukt. Dieses besagt, dass 

\begin{equation*}
    M^2 \cdot \frac{\lambda}{\pi} = \varphi \cdot \omega_0
\end{equation*}
wobei $\varphi$ der halbe Öffnungswinkel im Fernfeld, $\omega_0$ der Radius des Laserstrahls an seiner dünnsten Stelle,$M^2$ die Beugungsmaßzahl
und $\lambda$ die Wellenlänge ist.