\section{Laser}
\subsection{Funktionsweise eines Lasers}

Ein Laser ist ein Gerät, welches nach dem gleichnamigen Prozess, welches den Lichtstrahl des Lasers erzeugt. Das Herzstück des 
Lasers ist der optische Resonator (siehe Abb. \ref{bild:LaserAufbau}). Dieser ist im Allgemeinen ein Material, was die Energie des elektromagnetischen Feldes speichen kann. Dies 
geschieht meistens als stehende Welle, als wenn die Länge $L$ des Resonators 

\begin{equation*}
    L = N\cdot\frac{\lambda}{2}
\end{equation*}

ein vielfaches der halben Wellenlänge ist. Die Photonen wandern also in dieser stehende Welle durch den Resonator. Nun werden sie an den Enden
des Resonators von Spiegeln, hier konfokalen Spiegeln, reflektiert. \\
Damit die Photonen im Resonator mehr werden benötigt man die stimulierte – oft sagt man auch induzierte – Emission. Diese führt dazu, dass ein Photon 
bei einem Durchlauf ein weiteres Auslösen kann. Die für das Vermehren der Photonen benötigte Energie wird von der Pumpe geliefert. Diese kann optisch 
oder elektrisch funktionieren.\\

\begin{figure}[ht]
    \centering
    \includegraphics[width = 14cm]{Bilder/Auswertung/LaserAufbau.png}
    \caption{Elemente einen Lasers an unserem HeNe-Laser erklärt}
    \label{bild:LaserAufbau}
\end{figure}

Die Energie im elektromagnetischen Feld wird weniger durch Absorption im Resonator und durch das Austreten von Photonen an einem Spiegel.
Der Spiegel ist leicht durchlässig gebaut. Die Besonderheit des austretenden Lichtes ist, dass es im Phase,  vergleichsweise konstant in seiner Leistung, spektral 
schmalbandig und oft noch polarisiert ist. Das machen Laser ein gutes Mittel für die Anwendung im Labor. 

\clearpage
\subsection{Drei-/Vierniveaulaser}

Um den Laser zu realisieren muss das obere der am Strahlungsprozess beteiligte Niveau immer gut gefüllt sein. Man spricht
hierbei von einer Inversionslage. Um den Laserprozess aufrecht zu erhalten muss man weitere Hilfszustände
einführen. Bei einem Vierniveaulaser (siehe Abb. \ref{vrei}) hat man zwei weitere Hilfsniveaus, eines unter dem Grundzustand und eines über dem Grundzustand. 
Damit kann man einen Dauerstrichlaser aufrecht erhalten, da die Inverionslage aufrecht erhalten werden kann. \\
Ein Dreiniveaulaser (siehe Abb. \ref{drei}) hat nur ein weiters Hilffsniveau nämlich das unter dem Grundzustand. Bei beiden Lasern pumpt man auf
das höchste Energieniveau. 

\begin{figure}[ht]
    \centering
    \subfloat[Dreiniveaulaser]{\label{drei}%
    \includegraphics[width=0.35\textwidth]
    {Bilder/Auswertung/3NiveauLaser.png}}\quad
    \subfloat[Vierniveaulaser]{\label{vrei}%
    \includegraphics[width=0.25\textwidth]
    {Bilder/Auswertung/4NiveauLaser.png}}
      \caption{Drei- und Vierniveaulaser schematisch dargestellt \protect \footnotemark}
      \label{bild:LaserNiveaus}
\end{figure}

\footnotetext{Quelle: \url{http://www.pci.tu-bs.de/aggericke/PC4/Kap_III/Laser.htm} eingesehen 09.10.2021}

\subsection{Spontane und stimulierte Emission}

Bei der Emission gibt es zwei Arten, wie diese vonstatten gehen kann. Das zugrundeliegende Prinzip erklären wir an der spontanen Emission. Ein Photon wird emittiert, wenn ein Elektron 
von einem höheren in ein niedrigeres Energieniveau übergeht. Wie lagen der angeregte, höherenergetische Zustand bestehen bleibt lässt sich nicht vorhersagen. Man kann aber im Allgemeinen 
eine Energie-Zeit-Unschärfe Abschätzen, wie Heisenberg es getan hat, und ansetzten, dass 

\begin{equation*}
    \Delta E \cdot \Delta t \sim  \hbar
\end{equation*}

die Zeit invers Proportional zur Energie des angeregten Zustandes ist.\\
Die spontane Emission ist unabhängig von außeren Faktoren, wenn es darum geht, wann das Photon emittiert wird. Es muss
nur ein angeregtes Atom vorhanden sein. Die spontane Emission ist wichtig um den Laser zu starten. Von ihr stammt das erste Photon, was dann 
durch stimulierte Emission vervielfacht wird.\\
Bei der stimulierten Emission läuft der selbe Prozess ab, wie oben, jedoch wird der Übergang des Elektrons in ein niedrigeres Energieniveau
durch ein vorbeifliegendes Photon stimuliert. Das bei diesem Übergang freiwerdende Photon ist in Phase mit dem ersten Photon. Man hat also aus einem Photon
zwei Photonen gemacht, welche in Phase sind, weil in Abbildung \ref{bild:Emission} zu sehen.

\begin{figure} [ht] 
    \centering
    \includegraphics[width = 13cm]{Bilder/Auswertung/Emission.jpg}
    \caption{Prozess der spontanen Absorption, Emission und der stimulierten Emission\protect \footnotemark}
    \label{bild:Emission}
\end{figure}

\footnotetext{Quelle: \url{https://illumina-chemie.de/upload/30_66180837749063cd2437ad.jpg} eingesehen am 09.10.2021}

\subsection{Lasermoden}

Das Licht bildet in dem Resonator eine stehende Welle. Wie bei jeder stehenden Welle gibt es 
auch beim Laser Moden, hier transverale und axiale Moden.

\subsubsection{Transverale Moden}
Die Moden, welche einfach zu beobachten sind, sind die transversalen Moden. Diese sind durch eine Veränderung des
Strahlenprofils im Ausgang beobachtbar. Da die Symmetrie des Resonators durch die 
Ausrichtung der Brewsterfenster verschwindet, sieht man in x- und y-Richtung aufgespaltene Moden. Diese werden anhand der 
