\section{Schichtdickenmessung}

Die Schichtdickenmessung ist der umgekehrte Prozess zu dem in Kapitel \ref{section:Absorbtionskoeff} Versuchsteil. Diesmal hat man den linearen Absorptionskoeffizient gegeben und 
sucht die SChichtdicke.\\
Dazu misst man jeweils 4 Mal die Zahlrate unter einem Peak, jeweils mit und ohne Absorber dazwischen. Dies tut man, um die Präzsion der Messung zu überprüfen. Diese liegt im Rahmen der 
statistischen Schwankung. Die Position scheint keinen Einfluss auf die Messung zu haben und ist daher vernachlässigbar. Als Absorber wir hier Alufolie verwendet. Diese wird 
5 Mal gefaltet. Danaach hat man eine Absorberschicht aus 32 Lagen Alufolie. Die bei der Messung verwendete Gammastrahlenquelle ist Am-241, welche ihren Peak bei 0,0594 MeV hat. \\

Für die Schichtdicke wurden jeweils der Mittelwert der Zahlraten, welche proportional zur Intensität ist, aus den 4 Messungen gemittelt. Die Absorberdicke $d$ ergibt sich wie folgt:

\begin{equation}
    d = \frac{-\log{\frac{I}{I_0}}}{\mu}
\end{equation}

$I_0$ = Intensität vor Absorber, $I$ = Intensität mit Absorber, $\mu$ = linearer Absorptionskoeffizient (gegeben).

Nach klassischer Fehlerfortpflanzung ergibt sich:

\begin{center}
    \centering
    \textcolor{red}{$ d = (0,045\pm0,015) \mathrm{cm}$}\\
    \textcolor{red}{$ d_{Alufolie} = \frac{d}{32} = (0,00141\pm0,00050) \mathrm{cm}$}\\
\end{center}

Dabei muss man beachten, dass $d$ dann die Dicke aller Schichten bezeichnet. Also muss man, um aud die Dicke einer Lage Alu $d_{Alufolie}$ zu erhalten, 
$d$ noch durch 32 teilen.