%Matteo Kumar - Leonard Schatt
% Fortgeschrittenes Physikalisches Praktikum

% Teilauswertung X
\subsection{Fitten der Shockley-Gleichung}

Aus den Daten wurde mithilfe einens MatLab-Programmes ref die Parameter, welche in der Shockley-Gleichung vorkommen ref gefittet. Dabei wird beachtet, 
dass bei der CIS-Zelle einen Reihenschaltung von 11 kleinen Solarmodulen vorhanden ist. Daher wird die Spannung durch 11 geteilt. Leider erhält man 
trotzdem sehr fragwürdige Parameter, beispielsweise unrealistisch hohe Temperaturen als Umgebungsbedingungen hatten. Diese sind aber ausgeschlossen, 
da die Temperatur als Referenz gemessen wurde.
Interessant ist vorallem, dass sich mit dem Programm vor allem die Multi- und MonoSi besser mitder Shockley-Gleichung gefittet werden, als 
die CIS Solarzelle, was aber auch gut an der fehlerhaften Anwendung des Programmes oder dem Programm selbst.

\subsection{CIS-Solarmodul}

